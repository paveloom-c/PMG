\documentclass{beamer}
\special{pdf:minorversion 6}

\usepackage{fontspec}
\defaultfontfeatures{Ligatures={TeX}}
\setmainfont{cmun}[
  Extension=.otf,
  UprightFont=*rm,
  ItalicFont=*ti,
  BoldFont=*bx,
  BoldItalicFont=*bi,
]
\setsansfont{cmun}[
  Extension=.otf,
  UprightFont=*ss,
  ItalicFont=*si,
  BoldFont=*sx,
  BoldItalicFont=*so,
]
\setmonofont{cmun}[
  Extension=.otf,
  UprightFont=*btl,
  ItalicFont=*bto,
  BoldFont=*tb,
  BoldItalicFont=*tx,
]
\usepackage[main=russian,english]{babel}

\usepackage{amsmath}
\usepackage{amssymb}
\usepackage{gensymb}

\usepackage{hyperref}
\hypersetup{pdfstartview={FitH}}

\usepackage{booktabs}
\usepackage{makecell}

\usepackage[justification=justified, format=plain]{caption}
\usepackage{float}
\usepackage{graphicx}
\usepackage{subfig}
\graphicspath{ {../results/} }
\DeclareGraphicsExtensions{.pdf}
\captionsetup[table]{name=Табл.}
\captionsetup{labelsep=period}

\title{Пространственно-кинематическое моделирование плоской подсистемы Галактики методом наибольшего правдоподобия}
\subtitle{Выпускная квалификационная работа}
\author{Соболев Павел Леонидович\\Научный руководитель: Никифоров И. И.}

\date{6 июня 2023}

\setbeamertemplate{navigation symbols}{}
\setbeamertemplate{caption}[numbered]
\usefonttheme[onlymath]{serif}

\begin{document}

\begin{frame}
\titlepage
\end{frame}

\begin{frame}
\frametitle{Цель}
Корректность:
\begin{itemize}
  \item использование трехмерного поля скоростей;
  \item определение остаточного движения Солнца;
  \item учет вклада природной дисперсии;
  \item учет случайных ошибок гелиоцентрических расстояний;
  \item оптимизация порядка модели;
  \item использование гибкого алгоритма для определения выбросов по невязкам.
\end{itemize}
\end{frame}

\begin{frame}
\frametitle{Актуальность}
\begin{enumerate}
  \item Фундаментальность. От значения $ R_0 $ зависят, например:
  \begin{itemize}
    \item абсолютный размер нашей Галактики и её светимость;
    \item величина линейной скорости вращения Галактики $ \theta_0 $ на $ R_0 $;
    \item кривая вращения Галактики.
  \end{itemize}
  \item Вопрос получения достаточно точных оценок параметров. В последних обзорах средняя оценка $ R_0 $ расположена около 8 кпк.
\end{enumerate}
\end{frame}

\begin{frame}
\begin{figure}
  \centering
  \includegraphics[scale=1.15]{All/objects/Plots by type/XY (errors)}
  \caption{Распределение объектов каталога в плоскости Галактики}
\end{figure}
\end{frame}

\begin{frame}
\frametitle{Функция правдоподобия}
\vspace{-2em}
\begin{align}
  \nonumber
  L = \prod_{j = 1}^N & \frac{1}{\sqrt{2 \pi} (\sigma_{V_r})_j} \exp{\left\{ \frac{[ V_{r,j} - V_{r,\mathrm{mod}} (\varpi_{0,j}) ]^2}{2 (\sigma_{V_r})_j^2} \right\}} \times {}\\
  \nonumber
  {} \times {} & \frac{1}{\sqrt{2 \pi} (\sigma_{\mu_l'})_j} \exp{\left\{ \frac{[ \mu_{l,j}' - \mu_{l,\mathrm{mod}}' (\varpi_{0,j}) ]^2}{2 (\sigma_{\mu_l'})_j^2} \right\}} \times {}\\
  \nonumber
  {} \times {} & \frac{1}{\sqrt{2 \pi} (\sigma_{\mu_b})_j} \exp{\left\{ \frac{[ \mu_{b,j} - \mu_{b,\mathrm{mod}} (\varpi_{0,j}) ]^2}{2 (\sigma_{\mu_b})_j^2} \right\}} \times {}\\
  {} \times {} & \frac{1}{\sqrt{2 \pi} \sigma_{\varpi,j}} \exp{\left\{ \frac{(\varpi_j - \varpi_{0,j})^2}{2 \sigma_{\varpi,j}^2} \right\}}.
\end{align}
\end{frame}

\begin{frame}
\frametitle{Логарифмическая функция правдоподобия}
\vspace{-2em}
\begin{align}
  \nonumber
  \mathcal{L} &\equiv -\ln{L} = 4 N \ln{\sqrt{2 \pi}}\\
  \nonumber
  &+ \sum_{j=1}^N \left[ \ln{(\sigma_{V_r})_j} + \ln{(\sigma_{\mu_l'})_j} + \ln{(\sigma_{\mu_b})_j} + \ln{\sigma_{\varpi,j}} \right] + {}\\
  \nonumber
  &+ \frac{1}{2} \sum_{j=1}^N \min_{\varpi_{0,j}} \left\{ \frac{[ V_{r,j} - V_{r,\mathrm{mod}}(\varpi_{0,j}) ]^2}{(\sigma_{V_r})_j^2} + \frac{[ \mu_{l,j}' - \mu_{l,\mathrm{mod}}'(\varpi_{0,j}) ]^2}{(\sigma_{\mu_l'})_j^2} + {} \right.\\
  \nonumber
  &+ \left. \frac{[ \mu_{b,j} - \mu_{b,\mathrm{mod}}(\varpi_{0,j}) ]^2}{(\sigma_{\mu_b})_j^2} + \frac{( \varpi_j - \varpi_{0,j} )^2}{\sigma_{\varpi,j}^2} \vphantom{\frac{[ \mu_{l,j}' - \mu_{l,\mathrm{mod}}'(\varpi_{0,j}) ]^2}{(\sigma_{\mu_l'})_j^2}} \right\} = {}\\
  &= \mathcal{L}^{(0)} + \mathcal{L}^{(1)}(\mathbf{a}),
\end{align}
%
где $ \mathcal{L}^{(0)} = 4 N \ln{\sqrt{2 \pi}} + \sum_{j=1}^N \ln{\sigma_{\varpi,j}} = \mathrm{const} $, $ \mathbf{a} = (R_0, \omega_0, A, \theta_2, \ldots, \theta_n, u_\odot, v_\odot, w_\odot, \sigma_R, \sigma_\theta, \sigma_Z) $ --- вектор параметров модели. Решение --- точечная оценка вектора $ \mathbf{a} $ --- дается минимизацией функции $ \mathcal{L}^{(1)}(\mathbf{a}) $.
\end{frame}

\begin{frame}
\begin{figure}
  \centering
  \includegraphics[scale=0.75]{HMSFRs/n = 3/Inner profiles/29}
  \includegraphics[trim={0.51cm 0 0 0},clip,scale=0.75]{HMSFRs/n = 3/Inner profiles/61}
  \caption{Примеры профилей внутренней целевой функции, построенные по итоговым параметрам модели (выборка HMSFRs, $ n_\mathrm{o} = 3 $).}
\end{figure}
\end{frame}

\begin{frame}
\vspace{-0.6em}
\input{tables/hmsfrs.tex}
\end{frame}

\begin{frame}
\begin{figure}
  \centering
  \includegraphics[scale=1.27]{HMSFRs/n = 3/Fitted rotation curve (errors)}
  \caption{Модельная кривая вращения (выборка HMSFRs, $ n_\mathrm{o} = 3 $).}
\end{figure}
\end{frame}

\begin{frame}
\begin{figure}
  \centering
  \includegraphics[scale=1.1]{HMSFRs/n = 3/Conditional profile of R_0}
  \caption{Профиль целевой функции, построенный при фиксированном параметре $ R_0 $ и остальных свободных (выборка HMSFRs, $ n_\mathrm{o} = 3 $).}
\end{figure}
\end{frame}

\begin{frame}
\begin{figure}
  \centering
  \includegraphics[scale=0.79]{HMSFRs/L_1}
  \includegraphics[scale=0.79]{HMSFRs/sigma_theta}
  \caption{Зависимость значений целевой функции $ \mathcal{L}^{(1)} $ и азимутальной компоненты эллипсоида скоростей $ \sigma_\theta $ от порядка модели для выборки HMSFRs.}
\end{figure}
\end{frame}

\begin{frame}
\begin{figure}
  \centering
  \subfloat{{\includegraphics[scale=1.1]{HMSFRs/n = 3/Parallaxes (under 1 mas)}}}
  \caption{Соотношение исходных и приведенных параллаксов (в пределах 1 мсд, выборка HMSFRs, $ n_\mathrm{o} = 3$).}
\end{figure}
\end{frame}

\begin{frame}
\vspace{-0.6em}
\input{tables/hmsfrs_di.tex}
\end{frame}

\begin{frame}
\frametitle{Заключение}
\begin{enumerate}
  \item Создан новый каталог мазеров, содержащий наиболее полный на текущий момент набор кинематических данных (276 объектов).
  \item Разработан и реализован алгоритм пространственно-кинематического моделирования, учитывающий случайные ошибки гелиоцентрических расстояний.
  \item Выведена оценка расстояния от Солнца до центра Галактики $ R_0 = 7.92 \pm 0.12 $ кпк. Найдены статистически точные оценки для ряда фундаментальных параметров Галактики.
  \item Систематического смещения параллаксов не обнаружено: среднее смещение $ \Delta\varpi $ составляет $ −0.95 \pm 1.00 $ мксд. Неопределенности параллаксов мазеров завышены примерно на $ 1/5 $.
  \item Продемонстрирована необходимость учитывать неопределенность расстояний.
\end{enumerate}
\end{frame}

\begin{frame}
\centering\Huge
Спасибо за внимание!
\end{frame}

\end{document}
