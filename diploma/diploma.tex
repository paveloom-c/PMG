\documentclass[a4paper, oneside, 14pt]{article}
\special{pdf:minorversion 6}

\usepackage{geometry}
\newgeometry{vmargin=20mm,hmargin={30mm,20mm}}

\linespread{1.0}

\usepackage{fontspec}
\defaultfontfeatures{Ligatures={TeX}}
\setmainfont{cmun}[
  Extension=.otf,
  UprightFont=*rm,
  ItalicFont=*ti,
  BoldFont=*bx,
  BoldItalicFont=*bi,
]
\setsansfont{cmun}[
  Extension=.otf,
  UprightFont=*ss,
  ItalicFont=*si,
  BoldFont=*sx,
  BoldItalicFont=*so,
]
\setmonofont{cmun}[
  Extension=.otf,
  UprightFont=*btl,
  ItalicFont=*bto,
  BoldFont=*tb,
  BoldItalicFont=*tx,
]
\usepackage[main=russian,english]{babel}

\usepackage{amsmath}
\usepackage{amssymb}
\usepackage{gensymb}

\usepackage{tocloft}
\setcounter{tocdepth}{4}

\usepackage[bookmarks=false,hidelinks]{hyperref}
\hypersetup{pdfstartview={FitH}, pdfauthor={Павел Соболев}}

\hypersetup{pdftitle={Соболев (2023) - Пространственно-кинематическое моделирование плоской подсистемы Галактики методом наибольшего правдоподобия}}

\usepackage[table]{xcolor}
\usepackage{booktabs}
\usepackage{caption}

\usepackage{float}
\usepackage{subcaption}
\usepackage{graphicx}
\graphicspath{ {../plots/} }
\DeclareGraphicsExtensions{.pdf, .png}

\usepackage[
  backend=biber,
  autolang=other,
  sorting=ynt,
  style=phys,
  biblabel=brackets,
  maxbibnames=2,
  url=true,
]{biblatex}
\addbibresource{refs.bib}
\DeclareDelimFormat{finalnamedelim}{\addcomma\space}
\defbibheading{bibintoc}{\phantomsection\addcontentsline{toc}{section}{\bibname}\hyperlink{toc}{\section*{\bibname}}}
\AtEveryBibitem{
  \clearfield{urlyear}
  \clearfield{urlmonth}
}

\usepackage{indentfirst}

\usepackage{enumitem}
\setlist{nosep}

\newcommand{\su}{\vspace{-0.5em}}
\newcommand{\npar}{\par\vspace{\baselineskip}}

\makeatletter
\newcommand\subsubsubsection{\@startsection{paragraph}{4}{\z@}{-2.5ex\@plus -1ex \@minus -.25ex}{1.25ex \@plus .25ex}{\normalfont\normalsize\bfseries}}
\makeatother

\begin{document}

\begin{titlepage}
  \begin{center}
    {
      \Large
      \textbf{Санкт-Петербургский государственный университет}

      Математико-механический факультет\\
      Кафедра небесной механики\par
    }

    \vfill

    {
      \Large
      \textbf{Соболев Павел Леонидович}\\
      Выпускная квалификационная работа\par
    }

    \vspace{0.5cm}

    {
      \LARGE
      \textbf{Пространственно-кинематическое моделирование плоской подсистемы Галактики методом наибольшего правдоподобия}
    }

    \vspace{0.5cm}

    {
      \Large

      Уровень образования: специалитет\\
      Направление: 03.05.01 <<Астрономия>>\\
      Основная образовательная программа: СМ.5012.* <<Астрономия>>\par
    }

    \vfill

    \begin{flushright}
      \Large

      Научный руководитель:\\
      кандидат физико-математических наук,\\
      доцент кафедры небесной механики\\
      Никифоров И. И.\npar

      Рецензент:\\
      доктор физико-математических наук,\\
      главный научный сотрудник ГАО РАН\\
      Бобылев В. В.\par

    \end{flushright}

    \vfill

    {
      \Large
      Санкт-Петербург\\
      2023\par
    }
  \end{center}
\end{titlepage}

\begin{titlepage}
  \begin{center}
    {
      \Large
      \textbf{Saint-Petersburg State University}

      Mathematics \& Mechanics Faculty\\
      Chair of Celestial Mechanics\par
    }

    \vfill

    {
      \Large
      \textbf{Pavel Sobolev}\\
      Final qualification paper\par
    }

    \vspace{0.5cm}

    {
      \LARGE
      \textbf{Spatial-kinematic modelling of a flat subsystem of the Milky Way by the method of maximum likelihood}
    }

    \vfill

    \begin{flushright}
      \Large

      Scientific supervisor:\\
      Candidate of Sciences in Physical and Mathematical Sciences,\\
      Associate Professor of the Chair of Celestial Mechanics\\
      I. Nikiforov\npar

      Reviewer:\\
      Doctor of Sciences in Physical and Mathematical Sciences,\\
      Chief Researcher of the Central Astronomical Observatory of the Russian Academy of Sciences at Pulkovo\\
      V. Bobylev\par

    \end{flushright}

    \vfill

    {
      \Large
      Saint-Petersburg\\
      2023\par
    }
  \end{center}
\end{titlepage}

\hypertarget{toc}{}
\tableofcontents

\setcounter{page}{3}

\newpage

\setcounter{secnumdepth}{-1}
\hyperlink{toc}{\section{Введение}}

Настоящая выпускная квалификационная работа посвящена пространственно-кинематическому моделированию плоской подсистемы Галактики. Под этим подразумевается аналитическое построение кинематической модели Галактики и использование её для определения параметров вращения (постоянной Оорта $ A $, линейной скорости вращения на солнечном круге $ \theta_0 $ и др.) вкупе с определением масштабного параметра --- расстояния от Солнца до центра Галактики ($ R_0 $). Как следствие, сюда же включается и построение галактической кривой вращения. Результаты достигаются путем решения связанной с моделью задачи оптимизации на выборке мазерных источников с имеющимися данными об их положениях и скоростях.

Определение $ R_0 $ является фундаментальной проблемой, попытки решения которой предпринимались на протяжении уже более сотни лет. В \cite{N.2003} предложена классификация методов определения $ R_0 $ на пространственные, кинематические, динамические и нефазовые. Для каждого класса методов характерны различные источники систематических и случайных ошибок, оба варианта которых могут относиться как к определению ошибок расстояний до опорных объектов и других наблюдательных данных, так и к собственно методу определения $ R_0 $.

Первопроходцем в этом вопросе считается Харлоу Шепли (1918) \cite{S.1918}, обнаруживший сильную концентрацию шаровых скоплений вблизи центра Галактики (ЦГ). Расстояние от Солнца до точки наибольшей концентрации получилось равным $ \approx $ 13 кпк. Это был первый пространственный метод. В дальнейшем он модифицировался либо с целью устранения систематического эффекта селекции \cite{W.1975, F.W.1982, R.H.1989}, либо с целью его моделирования и учета \cite{R.P.D.F.1994}. Характерны также попытки устранить эффекты селекции другими методами: усечением Х-распределения шаровых скоплений вблизи галактического центра (Харрис, 1976) \cite{H.1976} и использованием конуса избегания (Сасаки и Исизава, 1978) \cite{S.I.1978}. С открытием звезд типа RR Лиры появился метод Бааде (1946, 1951) \cite{B.1946, B.1951}, заключающийся в исследовании вблизи ЦГ так называемых <<окон прозрачности>> и определении расстояния до точки наибольшей плотности в каждом из них. Сам Бааде получил оценку $ R_0 $ = 8.7 кпк \cite{B.1951}. Этот метод также применялся и для других типов объектов (мириды, карликовые цефеиды, звезды красного сгущения). Всё это пространственные методы, требующие учета искажений истинного распределения объектов и статистических эффектов.

Кинематические методы направлены на анализ кинематики галактических подсистем с целью локализации динамического ЦГ (барицентра). В отличие от пространственных, кинематические методы менее подвержены эффектам наблюдательной селекции, однако главной проблемой является выбор модели. Первым, кто использовал подобный метод для определения расстояния до ЦГ, по всей видимости, является Камм (1938) \cite{C.1938}. По данным о 15 планетарных туманностях он вывел оценку $ R_0 = 9.83 \pm 0.11 $ кпк. В дальнейшем кинематические методы получили широкое развитие и приобрели популярность. Согласно предложенной в \cite{N.2003} классификации, кинематические методы могут варьироваться по четырем признакам: по составу модели, по виду дифференциального вращения в модели, по способу оптимизации модели относительно $ R_0 $ и по учету неопределенности опорных расстояний. В данной работе в состав модели закладывается явление дифференциального вращения Галактики, предполагаемое осесимметричным, а также определяется остаточное движение Солнца; вид дифференциального вращения описывается многочленом Тейлора, порядок которого оптимизируется; находится самосогласованное решение для всей выборки опорных объектов (внутренняя оптимизация 1-ого рода); учитываются неопределенности опорных расстояний путем минимизации квадратов отклонений от модели одновременно и по положениям, и по скоростям.

\textbf{Актуальность} решения этой задачи остается высокой, ведь сама задача фундаментальна и лежит в основе многих вопросов галактической и внегалактической астрономии и астрофизики. В частности, от значения $ R_0 $ зависят \cite{N.2003} абсолютный размер нашей Галактики и её светимость, величина линейной скорости вращения Галактики $ \theta_0 $ на $ R_0 $, кривая вращения Галактики. Знание $ R_0 $ также позволяет понять природу галактического центра: размер, массу и светимость центральной области и населяющих её объектов. От значения $ \theta_0 $ (как и от значения постоянной Оорта A) зависит вид кривой вращения (в частности, является ли она в среднем убывающей, или примерно плоской, или в среднем возрастающей), что сильно влияет на динамические выводы. Сам закон вращения требуется для определения кинематических расстояний до объектов (при заданном $ R_0 $); исследования распределения масс в Галактике; моделирования динамических эффектов, возмущающих осесимметричное вращение Галактики (в частности, для моделирования спиральных рукавов). Учет остаточного движения Солнца является важным не только для повышения корректности кинематической модели, но и для решения таких сложных проблем, как определение радиуса коротации Галактики.

Благодаря осознанию важности определения основных галактических характеристик на совершенствование методов моделирования Галактики было затрачено много усилий, однако вопрос получения достаточно точных для использования в большинстве приложений оценок этих параметров до сих пор остается открытым. Это является второй причиной актуальности данной работы. Так, полученные с середины 70-х гг. XX века оценки $ R_0 $ охватывают интервал от 6 до 10.5 кпк \cite{K.L.1986, F.1987, R.1993}. На основании найденного в обзоре \cite{K.L.1986} среднего арифметического $ R_0 = 8.54 \pm 1.1 $ кпк по 25 оценкам $ R_0 $, полученным в 1974-1986 гг., 33-я комиссия МАС рекомендовала использовать новое стандартное значение $ R_0 = 8.5 $ кпк, сменившее старый стандарт МАС 1964 г. $ R_0 = 10 $ кпк (см. \cite{F.T.1991}). В более поздних обзорах \cite{R.1993, N.2004} средняя оценка $ R_0 $ расположена около 8 кпк. Со временем разброс уменьшается, но медленно. В работах 90-х годов ХХ века и более поздних оценки варьировались от 6.5 до 9 кпк.

\textbf{Цель} данной работы --- разработать наиболее корректный метод пространственно"=кинематического моделирования и применить его к данным о подсистеме опорных объектов. Корректность здесь достигается использованием трехмерного поля скоростей (включаются данные и о лучевых скоростях, и о собственных движениях); определением остаточного движения Солнца; учетом вклада природной дисперсии; учетом случайных ошибок гелиоцентрических расстояний; оптимизацией порядка модели (порядка полинома Тейлора при разложении кривой вращения); использованием гибкого алгоритма для определения выбросов по невязкам \cite{N.2012}.

В качестве опорных объектов используются мазеры. Опорные расстояния данных объектов определяются абсолютными методами (параллаксы определяются напрямую геометрическим методом c помощью радиоинтерферометрии со сверхдлинной базой), а значит, сравнительно менее подвержены систематическим ошибкам, характерным при использовании относительных методов (например, при использовании шкал расстояний). Плюсом также является точность измерений: для большей части имеющихся измерений были определены ошибки параллаксов и собственных движений, не превышающие соответственно 10\% и 1 мсд год$^{-1}$, а зависимость ошибки параллакса обратно пропорциональна гелиоцентрическому расстоянию \cite{N.V.2018}. Минусом применения данного типа объектов является относительная малочисленность данных: по сравнению с наборами данных для звезд красного сгущения (например, более 50 тысяч объектов в каталоге APOGEE-RC DR-17 \cite{DR17-APOGEE}), мазерных источников с полным набором кинематических данных на данный момент набирается чуть более пары сотен (хотя их число всё же увеличивается со временем). Автором данной работы объединены и дополнены каталоги VERA (2020) \cite{VERA.2020} и Рид и др. (2019) \cite{R.2019}.

\newpage

\setcounter{secnumdepth}{4}

\hyperlink{toc}{\section{Теория}}

\hyperlink{toc}{\subsection{Основные предположения и определения}}

Использование кинематического метода определения $ R_0 $ и других параметров предполагает составление модели подсистемы Галактики. С этим связана проблема зависимости результата от модельных и оптимизационных предположений. Некоторые из предположений делаются ради упрощения модели.\npar

\paragraph*{Предположения:}
\begin{enumerate}
\item Дифференциальное вращение Галактики является осесимметричным.
\item Остаточное движение Солнца может быть найдено в процессе оптимизации.
\item Глобальное расширение/сжатие Галактики не влияет на оценку $ R_0 $.
\item Влияние спиральной структуры Галактики \cite{B.O.1978} пренебрежимо.
\item К-член (член Кемпбелла, эмпирическая постоянная добавка в уравнениях для лучевой скорости) \cite{K.1978} полагается равным нулю.
\item Влияние эллиптической дисторсии галактического диска \cite{M.C.S.1998} пренебрежимо.
\item Оптимальный порядок $ n $ аппроксимирующего закон вращения полинома Тейлора может быть найден в ходе решения.
\end{enumerate}

\paragraph*{Определения:}
\begin{enumerate}
\item Модельная скорость заданного объекта --- скорость центроида объектов
данного типа, вычисленная для положения этого объекта.
\item Стандарт покоя подсистемы (СПП) --- гелиоцентрическая система отсчета, по отношению к которой среднее азимутальное движение объектов подсистемы, находящихся сейчас в области солнечного кольца (в окрестности солнечного круга $ R = R_0 $), равно нулю, а также равны нулю среднее движение вдоль координаты X и среднее вертикальное движение всех объектов подсистемы (рассматриваемой выборки). Иными словами, СПП в данную эпоху проходит через местоположение Солнца с азимутальной скоростью, равной средней азимутальной скорости объектов из области солнечного кольца, и с X- и Z-компонентами скорости, равными средним X- и Z-скоростям объектов всей выборки.
\end{enumerate}

\hyperlink{toc}{\subsection{Выражения для модельных величин наблюдаемых компонент 3D-скоростей}}

\hyperlink{toc}{\subsubsection{Лучевые скорости}}

В предположении чисто кругового вращения модельная величина гелиоцентрической лучевой скорости данного объекта, $ V_{r,\text{mod}} $, в общем случае определяется выражениями

\su\su
\begin{align}
  V_{r,\text{mod}} &= V_{r,rot} + V_{r,\odot},\\
  \label{eq:v_r_rot}
  V_{r,\text{rot}} &= (\omega - \omega_0) R_0 \sin{l} \cos{b},\\
  V_{r,\odot} &= -u_\odot \cos{l} \cos{b} - v_\odot \sin{l} \cos{b} - w_\odot \sin{b},
\end{align}

\noindent где $ V_{r,\text{rot}} $ и $ V_{r,\odot} $ --- влияние на модельную лучевую скорость объекта вращения подсистемы и движения Солнца относительно СПП, соответственно; $ \omega $ --- угловая скорость вращения подсистемы для точки, где находится рассматриваемый объект; $ \omega_0 $ --- угловая скорость СПП; $ u_\odot $, $ v_\odot $, $ w_\odot $ --- компоненты остаточной скорости Солнца, т.е. компоненты движения Солнца относительно СПП в направлениях $ (l, b) = (0\degree, 0\degree) $, $ (l, b) = (90\degree, 0\degree) $ и $ b = 90\degree $, соответственно; $ l $ и $ b $ --- галактические координаты объекта.

В случае плоской подсистемы линейная скорость вращения центроидов $ \theta = \theta(R) $. Используем для представления $ \theta = \theta(R) $ модельный полином в виде многочлена Тейлора:

\su
\begin{equation}
  \label{eq:model}
  \Theta_n(R) = \sum_{i=0}^n \frac{\theta_i}{i!} (\Delta R)^i, \quad n \ge 1, \quad \theta_i = \left. \frac{d^i \theta}{d R^i} \right|_{R = R_0},
\end{equation}

\noindent где

\su
\begin{equation}
  \Delta R \equiv R - R_0, \quad R = \sqrt{R_0^2 + r^2 \cos{b}^2 - 2 R_0 r \cos{l} \cos{b}}.
\end{equation}

\noindent Здесь $ R $ --- галактоосевое расстояние, $ r $ --- гелиоцентрическое расстояние до объекта.\par

Тогда модель \eqref{eq:v_r_rot} принимает общий вид

\su
\begin{equation}
  V_{r,\text{rot}} = \left[ -2 A \Delta R + \sum_{i=2}^n \frac{\theta_i}{i!} (\Delta R)^i \right] \frac{R_0}{R} \sin{l} \cos{b},
\end{equation}

\su
\begin{equation}
  A \equiv - \frac{1}{2} R_0 \omega'(R_0) = - \frac{1}{2} (\theta_1 - \omega_0).
\end{equation}

\hyperlink{toc}{\subsubsection{Собственные движения по долготе}}

Определение: $ \mu_l \equiv \frac{dl}{dt} $, $ \mu_l' \equiv \frac{dl}{dt} \cos{b} = \mu_l \cos{b} $. В обозначениях, аналогичных использованным в случае $ V_r $, для модельного полинома \eqref{eq:model}

\su\su
\begin{align}
  k \mu_{l,\text{mod}}' &= k \mu_{l,\text{rot}}' + k \mu_{l,\odot}',\\
  k \mu_{l,\text{rot}}' &= \left[ -2 A \Delta R + \sum_{i=2}^n \frac{\theta_i}{i!} (\Delta R)^i \right] \left( \frac{R_0 \cos{l}}{r} - \cos{b} \right) R^{-1} - \omega_0 \cos{b},\\
  k \mu_{l,\odot}' &= (u_\odot \sin{l} - v_\odot \cos{l}) / r,
\end{align}

\noindent где $ k = 4.7406 $ для $ r $ в кпк и $ \mu $ в мсд год$^{-1}$ \cite{G.N.O.2016}.

\hyperlink{toc}{\subsubsection{Собственные движения по широте}}

Определение: $ \mu_b \equiv \frac{db}{dt} $. Аналогично:

\su\su
\begin{align}
  k \mu_{b,\text{mod}} &= k \mu_{b,\text{rot}} + k \mu_{b,\odot},\\
  k \mu_{b,\text{rot}} &= \left[ 2 A \Delta R - \sum_{i=2}^n \frac{\theta_i}{i!} (\Delta R)^i \right] \frac{R_0}{R r} \sin{l} \sin{b},\\
  k \mu_{b,\odot} &= (u_\odot \cos{l} \sin{b} + u_\odot \sin{l} \sin{b} - w_\odot \cos{b}) / r,
\end{align}

\hyperlink{toc}{\subsection{Формулы редукции в случае мазеров}}

\hyperlink{toc}{\subsubsection{Переход от экваториальных сферических координат к галактическим}}

Если экваториальные координаты приведены на эпоху J2000.0, то $ l, b $ связаны с $ \alpha, \delta $ как

\su\su
\begin{align}
  \label{eq:lb_start}
  \sin{b} &= \sin{\delta} \cos{(90\degree - \delta_p)} - \cos{\delta} \sin{(\alpha - \alpha_p - 6^\text{h})} \sin{(90\degree - \delta_p)},\\
  l &= \phi + (\theta - 90\degree),\\
  \sin{\phi} &= \left[ \cos{\delta} \sin{(\alpha - \alpha_p - 6^\text{h})} \cos{(90\degree - \delta_p)} + \sin{\delta} \sin{(90\degree - \delta_p)} \right] / \cos{b},\\
  \label{eq:lb_end}
  \cos{\phi} &= \cos{\delta} \cos{(\alpha - \alpha_p - 6^\text{h})} / \cos{b},
\end{align}

\noindent где $ \alpha_p = 12^\text{h} 51^\text{m} 26^\text{s}.2817 $, $ \delta_p = 27\degree07'42''.013 $, $ \theta = 122\degree.932 $, $ \phi \in [0; 360) $ \cite{R.B.2004, R.2009}.

\hyperlink{toc}{\subsubsection{Преобразование собственного движения в экваториальных координатах в собственное движение в галактических координатах}}

В каталогах непосредственно приведены $ \mu_x = \mu_\alpha \cos{\delta} $ и $ \mu_y = \mu_\delta $ (\cite{R.2009} и последующие работы). Согласно рецепту Reid et al. (2009) \cite{R.2009}

\su
\begin{equation}
  \mu_l = l(\alpha + \mu_\alpha, \delta + \mu_\delta) - l(\alpha, \delta), \quad \mu_b = b(\alpha, \delta),
\end{equation}

\noindent где $ l(\alpha, \delta) $, $ b(\alpha, \delta) $ находятся по формулам \eqref{eq:lb_start}--\eqref{eq:lb_end}.

\hyperlink{toc}{\subsubsection{Восстановление гелиоцентрической скорости по скорости относительно Местного стандарта покоя}}

В каталоге приведены $ V_\text{LSR} $, вычисленные для стандартного движения Солнца $ V_\odot = 20 $ км/с в направлении $ \alpha(1900) = 18^\text{h} $, $ \delta(1900) = +30\degree $, обычно принимаемого в обсерваториях при обработке радионаблюдений \cite{R.2009}. Параметры этого движения:

\su
\begin{equation}
  u_\odot = 10.3 \text{ км/с}, \quad u_\odot = 15.3 \text{ км/с}, \quad w_\odot = 7.7 \text{ км/с}.
\end{equation}

\noindent Тогда

\su
\begin{equation}
  V_r = V_\text{LSR} - u_\odot \cos{l} \cos{b} - v_\odot \sin{l} \cos{b} - w_\odot \sin{b}.
\end{equation}

\hyperlink{toc}{\subsection{Преобразование лучевой скорости и собственного движения в линейные компоненты 3D-скорости для разных систем координат}}

\hyperlink{toc}{\subsubsection{Вычисление компонент скорости в галактических координатах}}

В общем случае

\su
\begin{equation}
  \label{eq:V_l_V_b}
  V_l \equiv r \mu_l' = r \mu_l \cos{b} = r \frac{dl}{dt} \cos{b}, \quad V_b = r \mu_b = r \frac{db}{dt}.
\end{equation}

Для компонент скорости, выраженных в километрах в секунду,

\su
\begin{equation}
  \label{eq:V_l_V_b_with_k}
  V_l \; [\text{км/с}] = k r \mu_l \cos{b} = k r \mu_l', \quad V_b \; [\text{км/с}] = k r \mu_b,
\end{equation}

\noindent где $ k = 4.7406 $ для $ r $ в кпк и $ \mu $ в мсд/год = mas y$^{-1}$, $ r $ [кпк] = $ 1 / \varpi $, где $ \varpi $ [мсд] -- параллакс, приведенный в каталоге.

\hyperlink{toc}{\subsubsection{Переход к компонентам скорости в декартовых гелиоцентрических координатах}}

Декартовы гелиоцентрические координаты:

\su\su
\begin{align}
  X &= r \cos{l} \cos{b},\\
  Y &= r \sin{l} \cos{b},\\
  Z &= r \sin{b}.
\end{align}

Тогда компоненты скорости в этих координатах выражаются следующим образом:

\su\su
\begin{align}
  U &\equiv \dot{X} = V_r \cos{l} \cos{b} - r \sin{l} \, \mu_l \cos{b} - r \cos{l} \sin{b} \, \mu_b,\\
  V &\equiv \dot{Y} = V_r \sin{l} \cos{b} + r \cos{l} \, \mu_l \cos{b} - r \sin{l} \sin{b} \, \mu_b,\\
  W &\equiv \dot{Z} = V_r \sin{b} + r \cos{b} \, \mu_b,
\end{align}

\noindent или, учитывая \eqref{eq:V_l_V_b},

\su\su
\begin{align}
  \label{eq:U}
  U &= (V_r \cos{b} - V_b \sin{b}) \cos{l} - V_l \sin{l},\\
  \label{eq:V}
  V &= (V_r \cos{b} - V_b \sin{b}) \sin{l} + V_l \cos{l},\\
  \label{eq:W}
  W &= V_b \cos{b} + V_r \sin{b}.
\end{align}

Для вычисления $ U $, $ V $ и $ W $, выраженных в километрах в секунду, в \eqref{eq:U}--\eqref{eq:W} следует подставлять значения $ V_l $ и $ V_b $, найденные по формулам \eqref{eq:V_l_V_b_with_k}.

\hyperlink{toc}{\subsubsection{Переход к галактоцентрической системе координат, связанной с Солнцем}}

Согласно формализму работы \cite{R.2009},

\su
\begin{equation}
  \label{eq:U_g_V_g_W_g}
  U_\text{g} = U + u_\odot, \quad V_\text{g} = V + \theta_\odot, \quad W_\text{g} = W + w_\odot,
\end{equation}

\noindent где используются значения компонент остаточного движения Солнца $ u_\odot $, $ w_\odot $ и линейной скорости вращения Солнца вокруг центра Галактики $ \theta_\odot = R_0 \omega_0 + v_\odot $, найденные при использовании уравнений раздела 1.1 по данным о скоростях и расстояниях, или, при отсутствии такого решения, подставляются результаты из \cite{R.2019}, полученные по мазерам: $ u_\odot = 10.7 $ км с$^{−1}$, $ \theta_\odot = 247 $ км с$^{−1}$, $ w_\odot = 7.7 $ км с$^{−1}$ (здесь $ u_\odot = u_\text{LSR} $ и $ w_\odot = w_\text{LSR} $ --- компоненты пекулярной скорости Солнца относительно МСП).

\hyperlink{toc}{\subsubsection{Переход к галактоцентрической системе координат, связанной с объектом}}

\hyperlink{toc}{\subsubsubsection{$\Lambda$-параметризация}}

Согласно формализму работы \cite{R.2009},

\su\su
\begin{align}
  \label{eq:Pi}
  \Pi &\equiv -\dot{R} = U_\text{g} \cos{\Lambda} - V_\text{g} \sin{\Lambda},\\
  \label{eq:theta}
  \theta &\equiv \dot{\Lambda} R = U_\text{g} \sin{\Lambda} + V_\text{g} \cos{\Lambda},\\
  \Upsilon &\equiv \dot{Z} = W_\text{g}.
\end{align}

Здесь $ \Lambda $ --- галактоцентрическая долгота объекта, которая отсчитывается от направления из центра Галактики на Солнце в сторону вращения Галактики:

\su
\begin{equation}
  \sin{\Lambda} = \frac{r \cos{b}}{R} \sin{l}, \quad \cos{\Lambda} = \frac{R_0 - r \cos{b} \cos{l}}{R};
\end{equation}

\noindent $ \Pi $, $ \theta $, $ \Upsilon $ --- радиальная (положительная к центру Галактики), азимутальная и вертикальная (положительная к северному полюсу Галактики) скорости объекта, соответственно. При отсутствии собственной оценки $ R_0 = 8.15 $ кпк \cite{R.2019}.

\hyperlink{toc}{\subsubsubsection{$\varphi$-параметризация}}

Вместо угла $ \Lambda $ можно ввести угол $ \varphi $ с вершиной в точке $ O' $ проекции объекта $ O $ на плоскость Галактики между направлением на центр Галактики и линией $ l = \text{const} $, $ b = 0\degree $ (между проекциями большой оси эллипсоида скоростей и луча зрения на плоскость Галактики) \cite{R.2017}. Угол $ \varphi $ отсчитывается против часовой стрелки, если смотреть с северного полюса Галактики. Очевидно, что $ \Lambda + l + \pi − \varphi = \pi $. Отсюда

\su
\begin{equation}
  \label{eq:lambda_varphi}
  \Lambda = \varphi - l, \quad \varphi = \Lambda + l.
\end{equation}

Тогда

\su\su
\begin{align}
  \label{eq:sin_phi}
  \sin{\varphi} &= \sin{\Lambda} \cos{l} + \cos{\Lambda} \sin{l},\\
  \label{eq:cos_phi}
  \cos{\varphi} &= \cos{\Lambda} \cos{l} - \sin{\Lambda} \sin{l},
\end{align}

\noindent или в силу \eqref{eq:lambda_varphi}

\su\su
\begin{align}
  \notag
  \sin{\varphi} &= \frac{r \cos{b}}{R} \sin{l} \cos{l} + \frac{R_0 - r \cos{b} \cos{l}}{R} \sin{l} =\\
  &= \frac{\sin{l}}{R} (r \cos{b} \cos{l} + R_0 - r \cos{b} \cos{l}) = \frac{R_0 \sin{l}}{R}.\\
  \notag
  \cos{\varphi} &= \frac{R_0 - r \cos{b} \cos{l}}{R} \cos{l} - \frac{r \cos{b}}{R} \sin^2{l} =\\
  &= \frac{1}{R} (R_0 \cos{l} - r \cos{b} \cos^2{l} - r \cos{b} \sin^2{l}) = \frac{R_0 \cos{l} - r \cos{b}}{R}.
\end{align}

$\Lambda$-параметризация удобнее при выражении галактоцентрических компонент скоростей через наблюдаемые компоненты. $\varphi$-параметризация, наоборот, удобнее при представлении наблюдаемых компонент через галактоцентрические.

\hyperlink{toc}{\subsection{Вклад эллипсоида скоростей в природные дисперсии измеренных компонент скорости объекта}}

\hyperlink{toc}{\subsubsection{Выражение компонент скорости $ V_r $, $ V_l $ и $ V_b $ через компоненты $ U $, $ V $ и $ W $}}

\hyperlink{toc}{\subsubsubsection{$ V_r $ в функции $ U $, $ V $ и $ W $}}

Используя \eqref{eq:W}, запишем:

\su
\begin{equation}
  \label{eq:V_b}
  V_b = \frac{W - V_r \sin{b}}{\cos{b}}.
\end{equation}

Комбинируя \eqref{eq:U} и \eqref{eq:V} и подставляя \eqref{eq:V_b}, получаем:

\su\su
\begin{gather}
  \nonumber
  U \cos{l} + V \sin{l} = V_r \cos{b} - V_b \sin{b} = V_r \cos{b} - \frac{W \sin{b}}{\cos{b}} + \frac{V_r \sin^2{b}}{\cos{b}} = \frac{V_r}{\cos{b}} - \frac{W \sin{b}}{\cos{b}}, \implies\\
  \label{eq:V_r}
  V_r = U \cos{l} \cos{b} + V \sin{l} \cos{b} + W \sin{b}.
\end{gather}

\hyperlink{toc}{\subsubsubsection{$ V_l $ в функции $ U $ и $ V $}}

Комбинируя \eqref{eq:U} и \eqref{eq:V}, получаем:

\su\su
\begin{gather}
  \nonumber
  U \sin{l} - V \cos{l} = -V_l \sin^2{l} - V_l \cos^2{l} = -V_l, \implies\\
  \label{eq:V_l}
  V_l = -U \sin{l} + V \cos{l}.
\end{gather}

\hyperlink{toc}{\subsubsubsection{$ V_b $ в функции $ U $, $ V $ и $ W $}}

Подставляя \eqref{eq:V_r} в \eqref{eq:V_b}, получаем:

\su\su
\begin{gather}
  \nonumber
  V_b = \frac{W}{\cos{b}} - U \cos{l} \sin{b} - V \sin{l} \sin{b} - W \frac{\sin^2{b}}{\cos{b}}, \implies\\
  \label{eq:V_b}
  V_b = -U \cos{l} \sin{b} - V \sin{l} \sin{b} + W \cos{b}.
\end{gather}

\hyperlink{toc}{\subsubsection{Дисперсии компонент скорости $ V_r $, $ V_l $, $ V_b $ и собственных движений $ \mu_l' $, $ \mu_b $ как функции дисперсий компонент $ U $, $ V $ и $ W $}}

Из выражений \eqref{eq:V_r}--\eqref{eq:V_b} в предположении отсутствия корреляции между $ U $, $ V $ и $ W $ по формуле распространения ошибок получаем следующие выражения для природных дисперсий линейных скоростей:

\su\su
\begin{align}
  \sigma_{V_r}^{\ast 2} &= \sigma_U^2 \cos^2{l} \cos^2{b} + \sigma_V^2 \sin^2{l} \cos^2{b} + \sigma_W^2 \sin^2{b},\\
  \sigma_{V_l}^{\ast 2} &= \sigma_U^2 \sin^2{l} + \sigma_V^2 \cos^2{l},\\
  \sigma_{V_b}^{\ast 2} &= \sigma_U^2 \cos^2{l} \sin^2{b} + \sigma_V^2 \sin^2{l} \sin^2{b} + \sigma_W^2 \cos^2{b}.
\end{align}

В силу \eqref{eq:V_l_V_b_with_k} $ \mu_l' = V_l / (k r) $. Тогда для дисперсий собственных движений имеем:

\su\su
\begin{align}
  \sigma_{\mu_l'}^{\ast 2} &= \frac{1}{k^2 r^2} (\sigma_U^2 \sin^2{l} + \sigma_V^2 \cos^2{l}),\\
  \sigma_{\mu_b}^{\ast 2} &= \frac{1}{k^2 r^2} (\sigma_U^2 \cos^2{l} \sin^2{b} + \sigma_V^2 \sin^2{l} \sin^2{b} + \sigma_W^2 \cos^2{b}).
\end{align}

\hyperlink{toc}{\subsubsection{Выражение компонент скорости $ V_r $, $ V_l $ и $ V_b $ через компоненты $ \Pi $, $ \theta $ и $ \Upsilon $}}

Комбинируя \eqref{eq:Pi} и \eqref{eq:theta}, запишем компоненты скорости $ U_\text{g} $, $ V_\text{g} $ и $ W_\text{g} $ как функции компонент $ \Pi $, $ \theta $ и $ \Upsilon $:

\su\su
\begin{align}
  \Pi \cos{\Lambda} + \theta \sin{\Lambda} &= U_\text{g} \cos^2{\Lambda} + U_\text{g} \sin^2{\Lambda} = U_\text{g},\\
  \Pi \sin{\Lambda} - \theta \cos{\Lambda} &= -V_\text{g} \sin^2{\Lambda} - V_\text{g} \cos^2{\Lambda} = -V_\text{g}.
\end{align}

Таким образом,

\su\su
\begin{align}
  \label{eq:U_g}
  U_\text{g} &= \Pi \cos{\Lambda} + \theta \sin{\Lambda},\\
  V_\text{g} &= -\Pi \sin{\Lambda} + \theta \cos{\Lambda},\\
  \label{eq:W_g}
  W_\text{g} &= \Upsilon.
\end{align}

\hyperlink{toc}{\subsubsubsection{В $\Lambda$-параметризации}}

Подставим \eqref{eq:U_g}--\eqref{eq:W_g} в \eqref{eq:V_r}, учитывая \eqref{eq:U_g_V_g_W_g}:

\su\su
\begin{align}
  \label{eq:V_r_Lambda}\nonumber
  V_r &= (U_\text{g} - u_\odot) \cos{l} \cos{b} + (V_\text{g} - \theta_\odot) \sin{l} \cos{b} + (W_\text{g} - w_\odot) \sin{b} =\\
  \nonumber
  &= (\Pi \cos{\Lambda} + \theta \sin{\Lambda} - u_\odot) \cos{l} \cos{b} + (-\Pi \sin{\Lambda} + \theta \cos{\Lambda} - \theta_\odot) \sin{l} \cos{b} + {}\\
  &+ (\Upsilon - w_\odot) \sin{b}.
\end{align}

Подставим \eqref{eq:U_g}--\eqref{eq:W_g} в \eqref{eq:V_l}, учитывая \eqref{eq:U_g_V_g_W_g}:

\su\su
\begin{align}
  \label{eq:V_l_Lambda}\nonumber
  V_l &= -(U_\text{g} - u_\odot) \sin{l} + (V_\text{g} - \theta_\odot) \cos{l} =\\
  &= -(\Pi \cos{\Lambda} + \theta \sin{\Lambda} - u_\odot) \sin{l} + (-\Pi \sin{\Lambda} + \theta \cos{\Lambda} - \theta_\odot) \cos{l}.
\end{align}

Подставим \eqref{eq:U_g}--\eqref{eq:W_g} в \eqref{eq:V_b}, учитывая \eqref{eq:U_g_V_g_W_g}:

\su\su
\begin{align}
  \label{eq:V_b_Lambda}\nonumber
  V_b &= -(U_\text{g} - u_\odot) \cos{l} \sin{b} - (V_g - \theta_\odot) \sin{l} \sin{b} + (W_g - w_\odot) \cos{b} =\\
  \nonumber
  &= -(\Pi \cos{\Lambda} + \theta \sin{\Lambda} - u_\odot) \cos{l} \sin{b} - (-\Pi \sin{\Lambda} + \theta \cos{\Lambda} - \theta_\odot) \sin{l} \sin{b} + {}\\
  &+ (\Upsilon - w_\odot) \cos{b}.
\end{align}

\hyperlink{toc}{\subsubsubsection{В $\varphi$-параметризации}}

Сгруппировав в \eqref{eq:V_r_Lambda}--\eqref{eq:V_b_Lambda} члены при $ \Pi $ и $ \theta $ и используя соотношения \eqref{eq:sin_phi}--\eqref{eq:cos_phi}, получаем

\su\su
\begin{align}
  \label{eq:V_r_phi}\nonumber
  V_r &= \Pi (\cos{\Lambda} \cos{l} - \sin{\Lambda} \sin{l}) \cos{b} + \theta (\sin{\Lambda} \cos{l} + \cos{\Lambda} \sin{l}) \cos{b} + \Upsilon \sin{b} - {}\\
  \nonumber
  &- u_\odot \cos{l} \cos{b} - \theta_\odot \sin{l} \cos{b} - w_\odot \sin{b} = {}\\
  &= \Pi \cos{\varphi} \cos{b} + \theta \sin{\varphi} \cos{b} + \Upsilon \sin{b} - u_\odot \cos{l} \cos{b} - \theta_\odot \sin{l} \cos{b} - w_\odot \sin{b},\\
  V_l &= -\Pi \sin{\varphi} + \theta \cos{\varphi} + u_\odot \sin{l} - \theta_\odot \cos{l},\\
  \label{eq:V_b_phi}
  V_b &= -\Pi \cos{\varphi} \sin{b} - \theta \sin{\varphi} \sin{b} + \Upsilon \cos{b} + u_\odot \cos{l} \sin{b} + \theta_\odot \sin{l} \sin{b} - w_\odot \cos{b}.
\end{align}

Здесь коэффициенты при $ \Pi $, $ \theta $ и $ \Upsilon $ соответствуют элементам матрицы $ G_S $ в \cite{R.2017}.

\hyperlink{toc}{\subsubsection{Дисперсии компонент скорости $ V_r $, $ V_l $, $ V_b $ и собственных движений $ \mu_l' $, $ \mu_b $ как функции дисперсий компонент $ \Pi $, $ \theta $ и $ \Upsilon $}}

\hyperlink{toc}{\subsubsubsection{В $\Lambda$-параметризации}}

Из выражений \eqref{eq:V_r_Lambda}--\eqref{eq:V_b_Lambda} в предположении отсутствия корреляции между $ \Pi $, $ \theta $ и $ \Upsilon $ по формуле распространения ошибок получаем соотношения для природных дисперсий линейных скоростей в функции дисперсий $ \mathbb{D} \Pi \equiv \sigma_R^2 $, $ \mathbb{D} \theta \equiv \sigma_\theta^2 $ и $ \mathbb{D} \Upsilon = \mathbb{D} W_g = \mathbb{D} W \equiv \sigma_Z^2 $:

\su\su
\begin{align}
  \label{eq:sigma_V_r_ast_Lambda}\nonumber
  \sigma_{V_r}^{\ast 2} = {} & \sigma_R^2 (\cos^2{\Lambda} \cos^2{l} \cos^2{b} - 2 \sin{\Lambda} \cos{\Lambda} \sin{l} \cos{l} \cos^2{b} + \sin^2{\Lambda} \sin^2{l} \cos^2{b}) + {}\\
  \nonumber
  {} + {} & \sigma_\theta^2 (\sin^2{\Lambda} \cos^2{l} \cos^2{b} + 2 \sin{\Lambda} \cos{\Lambda} \sin{l} \cos{l} \cos^2{b} + \cos^2{\Lambda} \sin^2{l} \cos^2{b}) + {}\\
  \nonumber
  {} + {} & \sigma_Z^2 \sin^2{b} = {}\\
  \nonumber
  {} = {} & \sigma_R^2 (\cos^2{\Lambda} \cos^2{l} - 2 \sin{\Lambda} \cos{\Lambda} \sin{l} \cos{l} + \sin^2{\Lambda} \sin^2{l}) \cos^2{b} + {}\\
  {} + {} & \sigma_\theta^2 (\sin^2{\Lambda} \cos^2{l} + 2 \sin{\Lambda} \cos{\Lambda} \sin{l} \cos{l} + \cos^2{\Lambda} \sin^2{l}) \cos^2{b} + \sigma_Z^2 \sin^2{b},\\
  \nonumber
  \sigma_{V_l}^{\ast 2} = {} & \sigma_R^2 (\cos^2{\Lambda} \sin^2{l} + 2 \sin{\Lambda} \cos{\Lambda} \sin{l} \cos{l} + \sin^2{\Lambda} \cos^2{l}) + {}\\
  {} + {} & \sigma_\theta^2 (\sin^2{\Lambda} \sin^2{l} - 2 \sin{\Lambda} \cos{\Lambda} \sin{l} \cos{l} + \cos^2{\Lambda} \cos^2{l}),\\
  \label{eq:sigma_V_b_ast_Lambda}\nonumber
  \sigma_{V_b}^{\ast 2} = {} & \sigma_R^2 (\cos^2{\Lambda} \cos^2{l} - 2 \sin{\Lambda} \cos{\Lambda} \sin{l} \cos{l} + \sin^2{\Lambda} \sin^2{l}) \sin^2{b} + {}\\
  {} + {} & \sigma_\theta^2 (\sin^2{\Lambda} \cos^2{l} + 2 \sin{\Lambda} \cos{\Lambda} \sin{l} \cos{l} + \cos^2{\Lambda} \sin^2{l}) \sin^2{b} + \sigma_Z^2 \cos^2{b}.
\end{align}

\hyperlink{toc}{\subsubsubsection{В $\varphi$-параметризации}}

Применяя к \eqref{eq:V_r_phi}--\eqref{eq:V_b_phi} формулу распространения ошибок в тех же предположениях или используя \eqref{eq:sigma_V_r_ast_Lambda}--\eqref{eq:sigma_V_b_ast_Lambda} с учетом соотношений \eqref{eq:sin_phi}--\eqref{eq:cos_phi} для $ \varphi $, находим выражения для дисперсий в этой параметризации:

\su\su
\begin{align}
  \sigma_{V_r}^{\ast 2} &= \sigma_R^2 \cos^2{\varphi} \cos^2{b} + \sigma_\theta^2 \sin^2{\varphi} \cos^2{l} + \sigma_Z^2 \sin^2{b},\\
  \sigma_{V_l}^{\ast 2} &= \sigma_R^2 \sin^2{\varphi} + \sigma_\theta^2 \cos^2{\varphi},\\
  \sigma_{V_b}^{\ast 2} &= \sigma_R^2 \cos^2{\varphi} \sin^2{b} + \sigma_\theta^2 \sin^2{\varphi} \sin^2{b} + \sigma_Z^2 \cos^2{b}.
\end{align}

Природные стандарты собственных движений находятся по общим формулам:

\su
\begin{equation}
  \sigma_{\mu_l'}^\ast = \frac{\sigma_{V_l}^\ast}{k r}, \quad \sigma_{\mu_b}^\ast = \frac{\sigma_{V_b}^\ast}{k r}.
\end{equation}

\hyperlink{toc}{\subsection{Алгоритм с учетом случайных ошибок гелиоцентрических расстояний}}

Пусть расстояния $ r $ определяются методом тригонометрического параллакса: $ r $ = $ 1 / \varpi $, где $ \varpi $ --- измеренное значение параллакса со средней ошибкой измерения $ \sigma_\varpi $. Тогда функция правдоподобия в случае моделирования поля 3D-скоростей является произведением четырех плотностей вероятностей, соответствующих нормально распределенным случайным величинам $ V_r $, $ \mu_l' $, $ \mu_b $ и $ \varpi $:

\su\su
\begin{align}
  \nonumber
  L = \prod_{j = 1}^N & \frac{1}{\sqrt{2 \pi} (\sigma_{V_r})_j} \exp{\left\{ \frac{\left[ V_{r,j} - V_{r,\text{mod}} (\varpi_{0,j}) \right]^2}{2 (\sigma_{V_r})_j^2} \right\}} \frac{1}{\sqrt{2 \pi} (\sigma_{\mu_l'})_j} \exp{\left\{ \frac{\left[ \mu_{l,j}' - \mu_{l,\text{mod}}' (\varpi_{0,j}) \right]^2}{2 (\sigma_{\mu_l'})_j^2} \right\}} \times {}\\
  {} \times {} & \frac{1}{\sqrt{2 \pi} (\sigma_{\mu_b})_j} \exp{\left\{ \frac{\left[ \mu_{b,j} - \mu_{b,\text{mod}} (\varpi_{0,j}) \right]^2}{2 (\sigma_{\mu_b})_j^2} \right\}} \frac{1}{\sqrt{2 \pi} \sigma_{\varpi,j}} \exp{\left\{ \frac{(\varpi_j - \varpi_{0,j})^2}{2 \sigma_{\varpi,j}^2} \right\}}
\end{align}

\noindent где $ V_{r,j} $, $ \mu_{l,j}' $, $ \mu_{b,j} $ и $ \varpi_j $ --- каталожные (измеренные) значения $ V_r $, $ \mu_l' $, $ \mu_b $ и $ \varpi $ для $j$-го объекта; $ \varpi_{0,j} $ --- приведенное значение параллакса $j$-го объекта. Итоговые дисперсии скоростных характеристик для $j$-го объекта представляют собой суммы двух составляющих:

\su\su
\begin{align}
  (\sigma_{V_r})_j^2 &= (\tilde{\sigma}_{V_r})_j^2 + (\sigma_{V_r}^\ast)_j^2,\\
  (\sigma_{\mu_l'})_j^2 &= (\tilde{\sigma}_{\mu_l'})_j^2 + (\sigma_{\mu_l'}^\ast)_j^2,\\
  (\sigma_{\mu_b})_j^2 &= (\tilde{\sigma}_{\mu_b})_j^2 + (\sigma_{\mu_b}^\ast)_j^2,
\end{align}

\noindent где $ \tilde{\sigma}_j^2 $ --- измерительная дисперсия, $ \sigma_j^{\ast 2} = \sigma_j^{\ast 2}(l_j, b_j, \varpi_{0,j}; \sigma_R, \sigma_\theta, \sigma_Z) $ --- вклад природной дисперсии. Приведенный параллакс рассматривается как нормально распределенная случайная величина $ \Pi_{0,j} \sim \mathrm{N}(\varpi_j, \sigma_{\varpi,j}^2) $. Тройка величин $ (l_j, b_j, \varpi_{0,j}) $ определяет точку (неортогональной) проекции объекта на кинематическую модель (кривую вращения) с учетом всех неопределенностей.\par

Тогда логарифмическая функция правдоподобия имеет вид

\su\su
\begin{align}
  \nonumber
  \mathcal{L} \equiv -\ln{L} &=
  \begin{aligned}[t]
    &-\sum_{j=1}^N \Biggl\{ -4 \ln{\sqrt{2 \pi}} - \ln{(\sigma_{V_r})_j} - \ln{(\sigma_{\mu_l'})_j} - \ln{(\sigma_{\mu_b})_j} - \ln{\sigma_{\varpi,j}} - {}\\
    &-\frac{\left[ V_{r,j} - V_{r,\text{mod}}(\varpi_{0,j}) \right]^2}{2 (\sigma_{V_r})_j^2} - \frac{\left[ \mu_{l,j}' - \mu_{l,\text{mod}}'(\varpi_{0,j}) \right]^2}{2 (\sigma_{\mu_l'})_j^2} - {}\\
    &-\frac{\left[ \mu_{b,j} - \mu_{b,\text{mod}}(\varpi_{0,j}) \right]^2}{2 (\sigma_{\mu_b})_j^2} - \frac{\left( \varpi_j - \varpi_{0,j} \right)^2}{2 \sigma_{\varpi,j}^2} \Biggr\} = {}
  \end{aligned}\\
  \nonumber
  &= 4 N \ln{\sqrt{2 \pi}}
  \begin{aligned}[t]
    &+ \sum_{j=1}^N \left[ \ln{(\sigma_{V_r})_j} + \ln{(\sigma_{\mu_l'})_j} + \ln{(\sigma_{\mu_b})_j} + \ln{\sigma_{\varpi,j}} \right] + {}\\
    &+ \frac{1}{2} \sum_{j=1}^N \min_{\varpi_{0,j}} \left\{ \frac{\left[ V_{r,j} - V_{r,\text{mod}}(\varpi_{0,j}) \right]^2}{(\sigma_{V_r})_j^2} + \frac{\left[ \mu_{l,j}' - \mu_{l,\text{mod}}'(\varpi_{0,j}) \right]^2}{(\sigma_{\mu_l'})_j^2} + {} \right.\\
    &+ \left. \frac{\left[ \mu_{b,j} - \mu_{b,\text{mod}}(\varpi_{0,j}) \right]^2}{(\sigma_{\mu_b})_j^2} + \frac{\left( \varpi_j - \varpi_{0,j} \right)^2}{\sigma_{\varpi,j}^2} \vphantom{\frac{\left[ \mu_{l,j}' - \mu_{l,\text{mod}}'(\varpi_{0,j}) \right]^2}{(\sigma_{\mu_l'})_j^2}} \right\} = {}
  \end{aligned}\\
  &= \mathcal{L}^{(0)} + \mathcal{L}^{(1)}(\mathbf{a}),
\end{align}

\noindent где $ L^{(0)} = 4 N \ln{\sqrt{2 \pi}} = \text{const} $, $ \mathbf{a} = (R_0, \omega_0, A, \theta_2, \ldots, \theta_n, u_\odot, v_\odot, w_\odot, \sigma_R, \sigma_\theta, \sigma_Z) $ --- вектор параметров модели.\par

Решение --- точечная оценка вектора $ \mathbf{a} $ --- для фиксированных модели и выборки дается минимизацией логарифмической функции правдоподобия:

\su
\begin{equation}
  \mathcal{L}^{(1)}(\mathbf{a}) \rightarrow \min.
\end{equation}

\newpage

\setcounter{secnumdepth}{-1}

\hyperlink{toc}{\section{Заключение}}

\newpage

\printbibliography[heading=bibintoc]

\end{document}
