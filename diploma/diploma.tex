\documentclass[a4paper, oneside, 14pt]{article}
\special{pdf:minorversion 6}

\usepackage{geometry}
\newgeometry{vmargin=20mm,hmargin={30mm,20mm}}

\linespread{1.0}

\usepackage[english, russian]{babel}

\usepackage{fontspec}
\setmainfont[
  Ligatures=TeX,
  Extension=.otf,
  BoldFont=cmunbx,
  ItalicFont=cmunti,
  BoldItalicFont=cmunbi,
]{cmunrm}
\usepackage{unicode-math}

\usepackage{tocloft}

\usepackage[bookmarks=false,hidelinks]{hyperref}
\hypersetup{pdfstartview={FitH}, pdfauthor={Павел Соболев}}

\hypersetup{pdftitle={Соболев (2023) - Пространственно-кинематическое моделирование плоской подсистемы Галактики методом наибольшего правдоподобия}}

\usepackage[table]{xcolor}
\usepackage{booktabs}
\usepackage{caption}

\usepackage{float}
\usepackage{subcaption}
\usepackage{graphicx}
\graphicspath{ {../plots/} }
\DeclareGraphicsExtensions{.pdf, .png}

\usepackage[
  backend=biber,
  sorting=ynt,
  style=phys,
  biblabel=brackets,
  maxbibnames=2,
  url=true,
]{biblatex}
\addbibresource{refs.bib}
\DeclareDelimFormat{finalnamedelim}{\addcomma\space}
\defbibheading{bibintoc}{\phantomsection\addcontentsline{toc}{section}{\bibname}\hyperlink{TOC}{\section*{\bibname}}}
\AtEveryBibitem{
  \clearfield{urlyear}
  \clearfield{urlmonth}
}

\usepackage{indentfirst}

\newcommand{\su}{\vspace{-0.5em}}
\newcommand{\npar}{\par\vspace{\baselineskip}}

\begin{document}

\begin{titlepage}
  \begin{center}
    {
      \Large
      \textbf{Санкт-Петербургский государственный университет}

      Математико-механический факультет\\
      Кафедра небесной механики\par
    }

    \vfill

    {
      \Large
      \textbf{Соболев Павел Леонидович}\\
      Выпускная квалификационная работа\par
    }

    \vspace{0.5cm}

    {
      \LARGE
      \textbf{Пространственно-кинематическое моделирование плоской подсистемы Галактики методом наибольшего правдоподобия}
    }

    \vspace{0.5cm}

    {
      \Large

      Уровень образования: специалитет\\
      Направление: 03.05.01 <<Астрономия>>\\
      Основная образовательная программа: СМ.5012.* <<Астрономия>>\par
    }

    \vfill

    \begin{flushright}
      \Large

      Научный руководитель:\\
      кандидат физико-математических наук,\\
      доцент кафедры небесной механики\\
      Никифоров И. И.\par

      \vspace{\baselineskip}

      Рецензент:\\
      доктор физико-математических наук,\\
      главный научный сотрудник ГАО РАН\\
      Бобылев В. В.\par

    \end{flushright}

    \vfill

    {
      \Large
      Санкт-Петербург\\
      2023\par
    }
  \end{center}
\end{titlepage}

\begin{titlepage}
  \begin{center}
    {
      \Large
      \textbf{Saint-Petersburg State University}

      Mathematics \& Mechanics Faculty\\
      Chair of Celestial Mechanics\par
    }

    \vfill

    {
      \Large
      \textbf{Pavel Sobolev}\\
      Final qualification paper\par
    }

    \vspace{0.5cm}

    {
      \LARGE
      \textbf{Spatial-kinematic modelling of a flat subsystem of the Milky Way by the method of maximum likelihood}
    }

    \vfill

    \begin{flushright}
      \Large

      Scientific supervisor:\\
      Candidate of Sciences in Physical and Mathematical Sciences,\\
      Associate Professor of the Chair of Celestial Mechanics\\
      I. Nikiforov\par

      \vspace{\baselineskip}

      Reviewer:\\
      Doctor of Sciences in Physical and Mathematical Sciences,\\
      Chief Researcher of the Central Astronomical Observatory of the Russian Academy of Sciences at Pulkovo\\
      V. Bobylev\par

    \end{flushright}

    \vfill

    {
      \Large
      Saint-Petersburg\\
      2023\par
    }
  \end{center}
\end{titlepage}

\hypertarget{TOC}{}
\tableofcontents

\setcounter{page}{3}

\newpage

\setcounter{secnumdepth}{-1}
\hyperlink{TOC}{\section{Введение}}

Настоящая выпускная квалификационная работа посвящена пространственно-кинематическому моделированию плоской подсистемы Галактики. Под этим подразумевается аналитическое построение кинематической модели Галактики и использование её для определения параметров вращения (постоянной Оорта $ A $, линейной скорости вращения на солнечном круге $ \theta_0 $ и др.) вкупе с определением масштабного параметра --- расстояния от Солнца до центра Галактики ($ R_0 $). Как следствие, сюда же включается и построение галактической кривой вращения. Результаты достигаются путем решения связанной с моделью задачи оптимизации на выборке мазерных источников с имеющимися данными об их положениях и скоростях.

Определение $ R_0 $ является фундаментальной проблемой, попытки решения которой предпринимались на протяжении уже более сотни лет. В \cite{N.2003} предложена классификация методов определения $ R_0 $ на пространственные, кинематические, динамические и нефазовые. Для каждого класса методов характерны различные источники систематических и случайных ошибок, оба варианта которых могут относиться как к определению ошибок расстояний до опорных объектов и других наблюдательных данных, так и к собственно методу определения $ R_0 $.

Первопроходцем в этом вопросе считается Харлоу Шепли (1918) \cite{S.1918}, обнаруживший сильную концентрацию шаровых скоплений вблизи центра Галактики (ЦГ). Расстояние от Солнца до точки наибольшей концентрации получилось равным $ \approx $ 13 кпк. Это был первый пространственный метод. В дальнейшем он модифицировался либо с целью устранения систематического эффекта селекции \cite{W.1975, F.W.1982, R.H.1989}, либо с целью его моделирования и учета \cite{R.P.D.F.1994}. Характерны также попытки устранить эффекты селекции другими методами: усечением Х-распределения шаровых скоплений вблизи галактического центра (Харрис, 1976) \cite{H.1976} и использованием конуса избегания (Сасаки и Исизава, 1978) \cite{S.I.1978}. С открытием звезд типа RR Лиры появился метод Бааде (1946, 1951) \cite{B.1946, B.1951}, заключающийся в исследовании вблизи ЦГ так называемых <<окон прозрачности>> и определении расстояния до точки наибольшей плотности в каждом из них. Сам Бааде получил оценку $ R_0 $ = 8.7 кпк \cite{B.1951}. Этот метод также применялся и для других типов объектов (мириды, карликовые цефеиды, звезды красного сгущения). Всё это пространственные методы, требующие учета искажений истинного распределения объектов и статистических эффектов.

Кинематические методы направлены на анализ кинематики галактических подсистем с целью локализации динамического ЦГ (барицентра). В отличие от пространственных, кинематические методы менее подвержены эффектам наблюдательной селекции, однако главной проблемой является выбор модели. Первым, кто использовал подобный метод для определения расстояния до ЦГ, по всей видимости, является Камм (1938) \cite{C.1938}. По данным о 15 планетарных туманностях он вывел оценку $ R_0 = 9.83 \pm 0.11 $ кпк. В дальнейшем кинематические методы получили широкое развитие и приобрели популярность. Согласно предложенной в \cite{N.2003} классификации, кинематические методы могут варьироваться по четырем признакам: по составу модели, по виду дифференциального вращения в модели, по способу оптимизации модели относительно $ R_0 $ и по учету неопределенности опорных расстояний. В данной работе в состав модели закладывается явление дифференциального вращения Галактики, предполагаемое осесимметричным, а также определяется остаточное движение Солнца; вид дифференциального вращения описывается многочленом Тейлора, порядок которого оптимизируется; находится самосогласованное решение для всей выборки опорных объектов (внутренняя оптимизация 1-ого рода); учитываются неопределенности опорных расстояний путем минимизации квадратов отклонений от модели одновременно и по положениям, и по скоростям.

\textbf{Актуальность} решения этой задачи остается высокой, ведь сама задача фундаментальна и лежит в основе многих вопросов галактической и внегалактической астрономии и астрофизики. В частности, от значения $ R_0 $ зависят \cite{N.2003} абсолютный размер нашей Галактики и её светимость, величина линейной скорости вращения Галактики $ \theta_0 $ на $ R_0 $, кривая вращения Галактики. Знание $ R_0 $ также позволяет понять природу галактического центра: размер, массу и светимость центральной области и населяющих её объектов. От значения $ \theta_0 $ (как и от значения постоянной Оорта A) зависит вид кривой вращения (в частности, является ли она в среднем убывающей, или примерно плоской, или в среднем возрастающей), что сильно влияет на динамические выводы. Сам закон вращения требуется для определения кинематических расстояний до объектов (при заданном $ R_0 $); исследования распределения масс в Галактике; моделирования динамических эффектов, возмущающих осесимметричное вращение Галактики (в частности, для моделирования спиральных рукавов). Учет остаточного движения Солнца является важным не только для повышения корректности кинематической модели, но и для решения таких сложных проблем, как определение радиуса коротации Галактики.

Благодаря осознанию важности определения основных галактических характеристик на совершенствование методов моделирования Галактики было затрачено много усилий, однако вопрос получения достаточно точных для использования в большинстве приложений оценок этих параметров до сих пор остается открытым. Это является второй причиной актуальности данной работы. Так, полученные с середины 70-х гг. XX века оценки $ R_0 $ охватывают интервал от 6 до 10.5 кпк \cite{K.L.1986, F.1987, R.1993}. На основании найденного в обзоре \cite{K.L.1986} среднего арифметического $ R_0 = 8.54 \pm 1.1 $ кпк по 25 оценкам $ R_0 $, полученным в 1974-1986 гг., 33-я комиссия МАС рекомендовала использовать новое стандартное значение $ R_0 = 8.5 $ кпк, сменившее старый стандарт МАС 1964 г. $ R_0 = 10 $ кпк (см. \cite{F.T.1991}). В более поздних обзорах \cite{R.1993, N.2004} средняя оценка $ R_0 $ расположена около 8 кпк. Со временем разброс уменьшается, но медленно. В работах 90-х годов ХХ века и более поздних оценки варьировались от 6.5 до 9 кпк.

\textbf{Цель} данной работы --- разработать наиболее корректный метод пространственно-кинематического моделирования и применить его к данным о подсистеме опорных объектов. Корректность здесь достигается использованием трехмерного поля скоростей (включаются данные и о лучевых скоростях, и о собственных движениях); определением остаточного движения Солнца; учетом вклада природной дисперсии; учетом случайных ошибок гелиоцентрических расстояний; оптимизацией порядка модели (порядка полинома Тейлора при разложении кривой вращения); использованием гибкого алгоритма для определения выбросов по невязкам \cite{N.2012}.

В качестве опорных объектов используются мазеры. Опорные расстояния данных объектов определяются абсолютными методами (параллаксы определяются напрямую геометрическим методом c помощью радиоинтерферометрии со сверхдлинной базой), а значит, сравнительно менее подвержены систематическим ошибкам, характерным при использовании относительных методов (например, при использовании шкал расстояний). Плюсом также является точность измерений: для большей части имеющихся измерений были определены ошибки параллаксов и собственных движений, не превышающие соответственно 10\% и 1 мсд год$^{-1}$, а зависимость ошибки параллакса обратно пропорциональна гелиоцентрическому расстоянию \cite{N.V.2018}. Минусом применения данного типа объектов является относительная малочисленность данных: по сравнению с наборами данных для звезд красного сгущения (например, более 50 тысяч объектов в каталоге APOGEE-RC DR-17 \cite{DR17-APOGEE}), мазерных источников с полным набором кинематических данных на данный момент набирается чуть более пары сотен (хотя их число всё же увеличивается со временем). Автором данной работы объединены и дополнены каталоги VERA (2020) \cite{VERA.2020} и Рид и др. (2019) \cite{Reid.2019}.

\newpage

\setcounter{secnumdepth}{2}

\hyperlink{TOC}{\section{Section}}

\hyperlink{TOC}{\subsection{Subsection}}

\newpage

\setcounter{secnumdepth}{-1}

\hyperlink{TOC}{\section{Заключение}}

\newpage

\printbibliography[heading=bibintoc]

\end{document}
